%----------------------------------------------------------------------------------------
%	RESEARCH EXPERIENCE SECTION
%----------------------------------------------------------------------------------------
\begin{rSection}{Research Experience}
    
    {\bf Optophysiology Lab, IMBIT//BrainLinks-BrainTools} \hfill {\em Aug. 2022 -- present}
    \\{\bf University of Freiburg, Germany}
    \\{\it Research Assistant} \hfill {\textit{\textbf{Supervisor:} Prof.\ Dr.\ Ilka Diester}}
    \begin{itemize}
        \item[-] Designed the Parallel Q-Learning with Hidden Markov Model (PQL-HMM) framework for the mathematical modeling of animal evidence based decision-making behavior.
        \item[-] Designed, implemented, and released a general reinforcement learning environment for rodent behavior simulation (Python module, available on GitHub).
    \end{itemize}
    
    {\bf Institute of Biology I \& Bernstein Center Freiburg} \hfill {\em July 2022 -- Nov. 2022}
    \\{\bf University of Freiburg, Germany}
    \\{\it Research Assistant} \hfill {\textit{\textbf{Supervisor:} Prof.\ Dr.\ Andrew D. Straw}}
    \begin{itemize}
        \item[-] Implemented a Tree of Parzen Estimators (TPE) method based parameter auto-tuning process of Kalman filter used for animal tracking.
        \item[-] Designed and implemented an event-camera-based lock-on tracker prototype, which is able to steer a multiple camera system for bee tracking in the wild. 
    \end{itemize}
        
    {\bf State Key Laboratory of Elemento-organic Chemistry} \hfill {\em Aug. 2017 -- Jan. 2021} 
    \\ {\bf Chemistry College, Nankai University, China}
    \\{\it Undergraduate Researcher / Research Assistant} \hfill {\textit{\textbf{Supervisor:} Prof.\ Dr.\ Xin Wen, Prof.\ Dr.\ Zhen Xi}}
    \begin{itemize}
        \item[-] Led National Training Program of Innovation and Entrepreneurship for Undergraduates: ``Computational Simulation and Biological Verification for Different Species of Protoporphyrinogen IX Oxidase Amino Acid Interactions''.
        \item[-] Completed bioinformatic analysis of multiple genera protoporphyrinogen oxidase (PPO) amino acid conservative property.
        \item[-] Constructed global dynamical amino acid interaction network for multiple genera PPO and mutants with data sampled from Molecular Dynamics (MD) simulation, with which further identified 67 potential key residues of \textit{human}PPO using graph algorithms and network theory.
        \item[-] Optimized our previous \textit{human}PPO mutant enzyme activity prediction method Prenzyme, tenfold increase in efficiency.
    \end{itemize}
    
    {\bf State Key Laboratory of Elemento-organic Chemistry} \hfill {\em Oct. 2016 -- Sep. 2017} 
    \\{\bf Chemistry College, Nankai University, China} 
    \\{\it Undergraduate Researcher} \hfill {\textit{\textbf{Supervisor:} Prof.\ Dr.\ Xuncheng Su}}
    \begin{itemize}
        \item[-] Synthesized an 1,3,4-oxadiazole-Based trifluoromethyl protein tag.
        \item[-] Conducted sampling and data analysis of $\mbox{}^1$H, $\mbox{}^{13}$C-NMR spectrum. 
    \end{itemize}
       
\end{rSection}
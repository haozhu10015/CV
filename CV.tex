\documentclass[12pt,a4paper]{moderncv}        % possible options include font size ('10pt', '11pt' and '12pt'), paper size ('a4paper', 'letterpaper', 'a5paper', 'legalpaper', 'executivepaper' and 'landscape') and font family ('sans' and 'roman')
\usepackage{lipsum}

\moderncvstyle{classic}                             % style options are 'casual' (default), 'classic', 'banking', 'oldstyle' and 'fancy'
\moderncvcolor{black}                               % color options 'black', 'blue' (default), 'burgundy', 'green', 'grey', 'orange', 'purple' and 'red'
\definecolor{color1}{RGB}{140, 21, 21}
\renewcommand{\familydefault}{\sfdefault}
\setlength{\hintscolumnwidth}{2cm}

\usepackage[utf8]{inputenc}
\usepackage{lmodern}
\usepackage[scale=0.85]{geometry}
\usepackage{fancyhdr}
\fancyfoot[R]{\thepage/\pageref{LastPage}}
\fancyfoot[L]{\textit{CV --- Hao Zhu}}
\usepackage{lastpage}
\usepackage{natbib}
\renewcommand{\bibsection}{}

\name{\textbf{Hao}}{\textbf{Zhu}}
\title{PhD student}
\address{\href{https://nr.informatik.uni-freiburg.de/people/hao-zhu}{Neurorobotics Lab}}{Georges-Koehler-Allee 201}{79110 Freiburg i. Br., Germany}
\email{zhuh@cs.uni-freiburg.de}
\extrainfo{\href{https://haozhu10015.github.io}{https://haozhu10015.github.io}}

\begin{document}
\makecvtitle

\section{\textbf{Education}}
\cventry{2024--}{PhD in Computer Science}{University of Freiburg}{Freiburg i. Br., Germany}{}{} 
\cvline{}{Advisors: Prof.~Joschka Boedecker}
\cventry{2021--2023}{M.Sc. in Neuroscience}{University of Freiburg}{Freiburg i. Br., Germany}{}{GPA - 3.8 $\mid$ German grade - 1.3} 
\cvline{}{Thesis: \textit{Deciphering Decision Making with Inverse Reinforcement Learning}}
\cvline{}{Advisors: Prof.~Ilka Diester, Prof.~Joschka Boedecker}
\cventry{2016--2020}{B.Sc. Chemical Biology}{Nankai University}{Tianjin, China}{}{GPA - 3.6 $\mid$ average grade 89.4/100} 
\cvline{}{Thesis: \textit{Identification of Functional Residues in the Human Protoporphyrinogen Oxidase with the Network Model and Site-Directed Mutagenesis}}
\cvline{}{Advisors: Prof.~Xin Wen, Prof.~Zhen Xi}

\section{\textbf{Academic Experience}}
\cventry{2024--}{Scientific Researcher}{\href{https://nr.informatik.uni-freiburg.de/welcome}{Neurorobotics Lab}, University of Freiburg}{Freiburg i. Br., Germany}{}{} 
\cvline{}{Part of the collaborative research project \textit{IN-CODE} on 1) fundamental research of deep learning and (inverse) reinforcement learning, and 2) application of machine learning in neuroscience.}

\cventry{2022--2023}{Research Intern}{\href{https://www.optophysiology.uni-freiburg.de/}{Optophysiology Lab}, University of Freiburg}{Freiburg i. Br., Germany}{}{} 
\cvline{}{Led a research project on the mathematical modeling of rodent complex foraging behavior via (inverse) reinforcement learning.}

\cventry{2022}{Research Intern}{\href{https://strawlab.org/}{Straw Lab}, University of Freiburg}{Freiburg i. Br., Germany}{}{} 
\cvline{}{
    Led a research project on 1) developing Kalman filter auto-tuning algorithm for animal tracking, and 2) designing event-camera-based lock-on tracker prototype, steering multiple cameras for tracking bees in the wild.
}

\cventry{2017--2021}{Undergraduate Researcher/Research Assistant}{\href{https://en-skleoc.nankai.edu.cn/}{State Key Laboratory of Elemento-organic Chemistry}, Nankai University}{Tianjin, China}{}{} 
\cvline{}{
    Led the National Training Program of Innovation and Entrepreneurship for Undergraduates project titled \textit{Computational Simulation and Biological Verification for Different Species of Protoporphyrinogen IX Oxidase Amino Acid Interactions}, aiming at identifying key amino acid residues in protoporphyrinogen oxidase with computational quantum mechanics, molecular dynamics simulation, and graph theory.
}

\section{\textbf{Awards}}
\cventry{2020}{Innovative Scientific Research Award for College Students}{Nankai University}{China}{}{Excellence Award}
\cventry{2018}{Asymchem Scholarship of Chemistry College}{Nankai University}{China}{}{}
\cventry{2017}{Asymchem Scholarship of Chemistry College}{Nankai University}{China}{}{}

\section{\textbf{Membership}}
\cvline{2024--}{IEEE}
\cvline{2024--}{IEEE Computational Intelligence Society}
\cvline{2024--}{German Neuroscience Society (GNS)}
\cvline{2024--}{Federation of European Neuroscience Societies (FENS)}

\section{\textbf{Teaching}}
\subsection{Lecture}
\cventry{2024}{Co-coordinator}{Probabilistic Graphical Models}{University of Freiburg}{}{}
\cvline{}{
    Developed the course materials, including lecture notes, slides, and exercises.
    Conducted all tutorials and co-lectured during lectures.
}

\cventry{2023}{Invited-lecturer}{Reinforcement Learning}{University of Freiburg}{}{}
\cvline{}{
    Invited talk about the application of inverse reinforcement learning methods in neuroscience and cognitive science.
}

\subsection{Seminar}
\cventry{2024}{Supervisor}{Causality and Reinforcement Learning}{University of Freiburg}{}{}
\cvline{}{Supervised students on the topic: causal discovery for time series data.}

\section{\textbf{Publications}}
\nocite*
\bibliography{papers}
\bibliographystyle{unsrtnat}

\section{\textbf{Services for Conferences and Workshops}}
\subsection{Reviewer}
\cventry{}{ICML 2024 Workshop AutoRL}{2024}{}{}{}

\section{\textbf{References}}
\begin{itemize}
    \item \textbf{Prof.~Joschka Boedecker}\\
        Department of Computer Science\\
        University of Freiburg\\
        Georges-Koehler-Allee 201\\
        79110 Freiburg i. Br., Germany\\
        +49 (0)761 203 8014\\
        \href{mailto: jboedeck@informatik.uni-freiburg.de}{jboedeck@informatik.uni-freiburg.de}\\
    \item \textbf{Prof.~Ilka Diester}\\
        IMBIT//BrainLinks-BrainTools\\
        University of Freiburg\\
        Georges-Koehler-Allee 201\\
        79110 Freiburg i. Br., Germany\\
        +49 (0)761 203 8440\\
        \href{mailto: ilka.diester@biologie.uni-freiburg.de}{ilka.diester@biologie.uni-freiburg.de}
\end{itemize}

\begin{center}
    \footnotesize
    \textit{Generated \today}
\end{center}

\end{document}
